\section{Advert Server Implementation}
\label{serverimpl}
Now that we designed major parts of our application storage server, we will
show some implementation details of our server.  

\subsection{Public Functions}
Each different client function is implemented as a different URL. This way we
have five URLs for different client functions. For example, \texttt{add()} is
implemented as \url{http://jondoe.appspot.com/add/} (where
\texttt{jondoe.appspot.com} is replaced by the user's own App Engine hostname).
Internally, the URLs are parsed by the \emph{webapp framework}
\cite{app-engine-webapp}, and the corresponding function is called. An example
is given in Figure \ref{serverimpl-webapp}.

\begin{figure*}[ht] %[placement] where placement is h,t,b,p
\begin{center}
\begin{code}
...
application = webapp.WSGIApplication(
                                     [('/',      MainPage),
                                      ('/add',   AddObject),
                                      ('/del',   DelObject),
                                      ('/get',   GetObject),
                                      ('/getmd', GetMetaData),
                                      ('/find',  FindMetaData)],
                                     debug=True)

def main():
  run_wsgi_app(application)

if __name__ == "__main__":
  main()
\end{code}
\caption{An Example of the webapp Framework.\label{serverimpl-webapp}}
\end{center}
\end{figure*}

All functions implemented are pretty straightforward, or otherwise
explained in the other sections below. The only two functions that are not
exlpained are the \texttt{delete()} and the \texttt{find()} functions. We will
explain both functions next.

\subsubsection{Find}
\label{serverimpl-find}
When the meta data has been stored in the datastore, we should be able to process
\texttt{find()} requests by clients. This is the most complex function of the
advert advert server. First a meta data object is received, which is then tried
to be matched with meta data already present in the datastore. For this purpose
we will use the GQL as described in Section \ref{serverdesign-gql}.

To process the \texttt{find()} request, we have to check all the key-value pairs
given by the client and compare them to the key-value pairs of all MetaData objects
in the datastore. Since all key-value pairs are stored in one `table', we
have to group them by path, after which we can check all key-value pairs per
meta data object.

Comparing the meta data object sent by the user with the meta data present in
the datastore can be achieved in two ways. One approach is to make a selection of
all paths available and check the key-value pairs on a per path basis. This
is done by the GQL query shown in Figure \ref{serverimpl-find-func}.

\begin{figure*}[ht] %[placement] where placement is h,t,b,p
\begin{center}
\begin{code}
query = db.GqlQuery("SELECT * FROM MetaData")

paths = Set()

for bin in query:
  paths.add(bin.path)
  
paths = list(paths)

for path in paths[:]:
  for k in json.keys():
    query = db.GqlQuery("SELECT * FROM MetaData WHERE path = :1 AND keystr = :2
                         AND val = :3", path, k, json[k])
    if query.count() < 1:
      paths.remove(path)
      break

...
\end{code}
\caption{\texttt{find()} Function in Detail.\label{serverimpl-find-func}}
\end{center}
\end{figure*}

To obtain an array of all paths (without duplicates), we would usually use the
`\texttt{DISTINCT}' keyword from SQL. Since the GQL does not support the
`\texttt{DISTINCT}' keyword we have to make our own unique array of keys. We do
this by using the \texttt{Set()} data type, which maintains a unique set of paths
(duplicates are automatically overwritten).

After we have a distinct set of all paths, we iterate through the list and match
every key-value pair with the key-value pair given by the user. If one pair does
not match, we remove the path from the list, and start over inspecting the next
path in the list. Eventually we will have a list of paths which contains all
paths that meet all requirements given by the user.

The return value of all meta data found is a String array, which again needs to
be formatted in such a way that it can be sent as one String. Obviously we will
use \emph{simplejson} (a Python implementation of JSON, see Section
\ref{serverimpl-simplejson}) to do so. If no entry is found, we will send an
error to user, stating that no matching meta data was found.

A second option to implement the \texttt{find()} function would be to first find
all paths that meet the requirements given by the first key-value pair. Then do
the same for the second key-value pair and accordingly take the intersection
between the two search results. This process should be repeated until all
key-value pairs have been checked. The downside of this approach is that there is
no obvious method to perform the intersection between two search results at the
Google App Engine. This approach requires a \emph{for()}-loop to iterate through
all items in the first search result and check them with the second.

\subsubsection{Delete}
To remove an (Advert) object from the datastore, a user calls the
\texttt{delete()} function, with a pathname as argument. Obviously, the most
straightforward way to remove an item from the datastore is to query all objects
and meta data that match the pathname given. Naturally, we want to perform the
removal of the object and associated meta data in transaction (see Section
\ref{serverimpl-transactions}, because we do not want inconsistencies in our
datastore (i.e. querying meta data of which the associated object does not exist
anymore).

The only problem we have now is that we cannot perform queries inside a
transaction, since the Google App Engine does not allow us to do so. The
solution that Google provides is either to work with keys (which is not
feasible in our case), or prepare your queries before performing the
transaction. We adapted this solution by sending our query results to a
\texttt{remove()} function, which iterates through the query results and
deletes all objects and meta data given to the function.

\subsection{Datastore}
In this section we describe how data storage is done in more detail than
described in our design (Section \ref{serverdesign-datastore}).

\subsubsection{Storing Data}
Data is added to the datastore using the \texttt{add()} function. This function
contains three function arguments: the actual data to be added, additional meta
data to be added, and a path name which identifies the data. Once all arguments
have been decoded from the HTTP message body (see Section
\ref{serverimpl-simplejson}), we are ready to store the data.

Note that an Advert object can be \texttt{null}. Although we do not see why just
storing meta data would be useful, we do allow to store a \texttt{null} object.
Just like the Advert object, a meta data object can also be \texttt{null} (i.e.
no meta data is to be stored). The only way the object can then be referenced is
by using the specified pathname.

\subsubsection{Transactions}
\label{serverimpl-transactions}
Since the (Advert) object and the meta data object are stored sequentially, it
is essential that they are stored either both, or neither of them, to ensure
consistency. Fortunately, the Google App Engine datastore provides a mechanism
for ensuring both are stored or, if the datastore fails, nothing is stored.
Using transactions, we can call a function and maintain atomicity. All
functionality of the App Engine can be used inside a function that is called in
transaction, except for GQL functions.

Another prequisite of transactions is that all data operated on must be in the
same entity group (this includes retrieving entities by key, updating entities, and
deleting entities). Entity group relationships tell App Engine to store several
entities in the same part of the distributed network. When the application
creates an entity, it can assign another entity as the parent of the new entity,
using the parent argument in the Model constructor. Assigning a parent to a new
entity puts the new entity in the same entity group as the parent entity. This is
exactly what we do when storing an object with meta data attached to it (all
meta data which belongs to a certain object have the same parent, being the
object itself). Notice that each root entity belongs to a separate entity group,
so a single transaction cannot create or operate on more than one root entity. 

\begin{figure*}[ht] %[placement] where placement is h,t,b,p
\begin{center}
\begin{code}
try:
  db.run_in_transaction(store, json) 
except db.TransactionFailedError, message:
  logging.error(message)
  ...
\end{code}
\caption{Transactions.\label{serverimpl-trans-func}}
\end{center}
\end{figure*}

A small example of how transactions work inside the App Engine is shown in Figure
\ref{serverimpl-trans-func}. In this example we encapsulated the transaction
function \texttt{store}, with arguments \texttt{json}, into a \texttt{try},
\texttt{except} statement. If the transaction fails, the operation's effects are not
applied, and the datastore API raises an exception, which in turn is caught by
the \texttt{except} statement and an error message is attached. The transction
could fail due to a high rate of contention, with too many users trying
to modify an entity at the same time. Or an operation may fail due to the
application reaching a quota limit. Or there may be an internal error with the
datastore.

Note that to ensure consistency of the datastore, we will also use transactions
to remove objects from the datstore. Again this is done sequentially. If one
delete function would fail, we would end op with an inconsistent copy of the
datastore. Hence we will again use transactions for atomicity and thus
consistency.

\subsubsection{Path Encoding}
To have the AdvertService work properly, we need some path encoding to save
Advert objects in the datastore. The Google App Engine itself provides one
solution, because every entity in the datastore has a key. When the application
creates an entity, it can assign another entity as the parent of the new entity.
Assigning a parent to a new entity puts the new entity in the same entity group
as the parent entity. Every entity belongs to an entity group, a set of one or
more entities that can be manipulated in a single transaction. An entity without
a parent is a root entity. An entity that is a parent for another entity can also
have a parent. A chain of parent entities from an entity up to the root is the
path for the entity, and members of the path are the entity's ancestors. The
parent of an entity is defined when the entity is created, and cannot be changed
later. The downside of this solution is that we have to create empty MetaData
objects for path nodes that are empty.

Another option is to save our own paths, using a namespace that identifies a
path, something like the UNIX system does (e.g. \texttt{/home/bboterm/.}). The
advantage of this model is that it works exactly analogue to the JavaGAT
AdvertService. The disadvantage of this is that we have to parse our own
pathnames, and keep track of them by means of a non-existing mechanism (i.e.
something that we need to build ourselves). Since this is not a major obstacle,
we will stick with the second approach. Note that all pathnames are allowed at
the server side. They are just used as identifiers of objects, not actual paths.

\subsubsection{Garbage Collection}
As described above, we will use some form of TTL to determine whether a data item
stored is still needed in the datastorage. When a data item is expired, it will
be removed automatically using a mechanism called the garbage collector.
The source code of our garbage collector can be found in Figure
\ref{serverimpl-gc}.

\begin{figure*}[ht] %[placement] where placement is h,t,b,p
\begin{center}
\begin{code}
def gc(): #garbage collector
  query = db.GqlQuery("SELECT * FROM Advert WHERE ttl < :1", 
                      datetime.datetime.today() + datetime.timedelta(days=-10))
  
  for advert in query: #all entities that can be deleted
    advert.delmd()     #delete all associated metadata
    advert.delete()    #delete the object itself
\end{code}
\caption{The Garbage Collector.\label{serverimpl-gc}}
\end{center}
\end{figure*}

By calling \texttt{gc()}, we activate our garbage collector, which, in turn,
executes a GQL query, selecting all data that has been added to the datastore
earlier than ten days ago. To achieve this we use the \texttt{timedelta()}
object, provided by the \texttt{datetime} class. A timedelta object
represents a duration, the difference between two dates or times, and is ideal
for our garbage collector.

The query results in an iterable list of advert objects that can be removed
from the datastore. We should not forget that also all the MetaData
associated with the Advert object should be removed alongside the object itself.
Therefore we wrote a function inside the advert class, called \texttt{delmd()},
that automatically deletes all its metadata. Once this is done, the object
itself is deleted and the garbage collection process is finished.

\subsection{Transfer Protocol}
Below we describe our transfer protocol in more details, starting with 
      
\subsubsection{Receiving and Storing Binary Data}
An important function of the advert server is to accept and store binary data.

\paragraph{Binary Data Only}
For accepting just binary data, we can receive the entire message body and write
it to a variable, which is stored in the datastore accordingly. A sample code of
this function is shown in the code segment of Figure \ref{serverimpl-download}.
The first line identifies the class, which is called by the RequestHandler
(depending on the given URL). The second line says that we are expecting a POST
request. Then, a new \texttt{Bin()} data model is created (which in our case only
contains a field \texttt{data = db.BlobProperty()}). Then we fetch the body and
put in a temporary variable, before we store it as a \texttt{db.BlobProperty()}in
the database by calling \texttt{bin.put()}.

\begin{figure*}[ht] %[placement] where placement is h,t,b,p
\begin{center}
\begin{code}
class Download(webapp.RequestHandler):
  def post(self):
    bin = Bin()
    uploaded_file = self.request.body
    bin.data = db.Blob(uploaded_file)
    bin.put()
    self.redirect('/')
\end{code}
\caption{Accepting Binary Data.\label{serverimpl-download}}
\end{center}
\end{figure*}
      
\paragraph{Combination of Binary Data and Strings}
Although meta data will be stored in a different class, we will not apply a
different function to store meta data objects, for the simple reason that if one
of these transfers would fail, it could lead to inconsistencies (see Section
\ref{serverimpl-transactions}).

When receiving our binary object as a combination of both binary data and
unicode Strings, it will be received as one stream of bytes (assuming we are not
using the multipart/formdata method of Section \ref{serverdesign-encoding}.
This means that we will have to dismember the message ourselves, which is
not as straightforward as it seems. The Google App Engine currently only
supports Python 2.5, which does not support bytes or byte arrays for data
manipulation.

We still managed to extract data using string manipulation. Essentially when the
App Engine receives the entire body as shown in Figure \ref{serverimpl-download},
it is stored as a raw string (no encoding) before it is stored as BLOB in the
datastore. We made use of that raw string as shown in Figure
\ref{serverimpl-raw-string}. In this example, we receive the message body and
store it into the raw string called `bytes'. After this we read the first byte
(\texttt{bytes[0:4]}), we use \texttt{ord()} to make it an integer, knowing the
length of the data sent. After that we send our Content-Type headers and write
the rest of the body to the standard output.

\begin{figure*}[ht] %[placement] where placement is h,t,b,p
\begin{center}
\begin{code}
bytes = self.request.body
length = ord(bytes[0:4])
self.response.headers['Content-Type'] = "image/gif"
self.response.out.write(bytes[5:length])
\end{code}
\caption{Manipulating a Raw String.\label{serverimpl-raw-string}}
\end{center}
\end{figure*}

\paragraph{Simplejson}
\label{serverimpl-simplejson}
Instead of using our own representation of unicode Strings and binary data, we
could also use a JSON representation. Python has its own implementation of
JSON called \emph{simplejson} \cite{simplejson-www}, which is a
simple, fast, complete, correct and extensible JSON encoder and decoder for
Python 2.4+. It is pure Python code with no dependencies, but includes an
optional C extension for a serious speed boost. From Python 2.6, JSON is
contained in the standard library, but since Google makes use of Python version
2.5.2, we use simplejson as provided by Django. 

Once a serial JSON array or object has been sent (see Section
\ref{clientimpl-encoding}), we are able to load it into a server-side JSON
object and decode it according to Figure \ref{serverimpl-json}. The
\texttt{simplejson.loads()} function decodes a serial JSON String representation
into a JSON array, after which we can access its array entries like a regular
array.

\begin{figure*}[ht] %[placement] where placement is h,t,b,p
\begin{center}
\begin{code}
body = self.request.body
json = simplejson.loads(body)
...
advert.path   = json[0] #extract path from message
advert.object = json[2] #extract (base64) object from message
...
for k in json[1].keys():
  ...
  metadata.keystr = k
  metadata.value  = json[1][k]
...
\end{code}
\caption{Decoding a JSON object.\label{serverimpl-json}}
\end{center}
\end{figure*}

The same goes for a JSON object. We can construct a for loop like shown in Figure
\ref{serverimpl-json}, by iterating through all the keys and retrieving their
key-value pairs. The only option is to use the \texttt{db.TextProperty()}, since
it is not limited to 500\,characters, like the \texttt{db.StringProperty()} is.

Before storing the JSON object it is checked to be a valid JSON object, which
should adhere to the syntax shown in Figure \ref{serverimpl-json-syntax}.
Otherwise an error is returned and the object is not stored. Note that storing a
JSON object is done in an atomic transaction, as described in section
\ref{serverimpl-transactions}.

Once data has been stored successfully, we send an HTTP 201 (Created) status
code, and return the TTL in the message body. If an entry already exists at the
specified path, that entry gets overwritten, and a warning is issued. This is
done by sending an HTTP 205 (Reset Content) Status Code and a warning in the
message body.

\begin{figure*}[ht] %[placement] where placement is h,t,b,p
\begin{center}
\begin{code}
[ string , { string : string , string : string , ... } | null , string | null ]
\end{code}
\caption{Advert syntax of a JSON object.\label{serverimpl-json-syntax}}
\end{center}
\end{figure*}

% \subsubsection{Storing Meta Data}
% Note that if meta data is \texttt{null}, no meta data will be stored in the
% datastore, since it would not make any sense to search for data that is never
% sent.

\subsubsection{Returning Data From the Datastore}
As long as we only have to send one item from the datastore to the user, we do
not need to do much encoding. We just have to setup the content-type, which is
plain text (i.e. unicode String), and send the actual data (see Figure
\ref{serverimpl-general-response}. This is the case when the user calls the
\texttt{get()} function. Of Course, we first need to fetch the requested object
from the datastore, which will be done using GQL Queries.

\begin{figure*}[ht] %[placement] where placement is h,t,b,p
\begin{center}
\begin{code}
self.response.headers['Content-Type'] = "text/plain"
self.response.out.write(object.data)
\end{code}
\caption{HTTP Response.\label{serverimpl-general-response}}
\end{center}
\end{figure*}

In addition, a user can also call more complex functions than the
\texttt{get()} function. For example the \texttt{getmd()} and \texttt{find()}
(see Section \ref{serverimpl-find}) function have more complex results than the
\texttt{get()} function. Both functions need to send multiple data items in one
HTTP response, which is also done using JSON encoding. Once all the data has been
fetched using GQL Queries, the data is added to a JSON object or array,
respectively, and serialized (\texttt{simplejson.dumps()}) before it is returned
just like Figure \ref{serverimpl-general-response}. An example of how this is
done can be seen in Figure \ref{serverimpl-json-response}.

\begin{figure*}[ht] %[placement] where placement is h,t,b,p
\begin{center}
\begin{code}
query = db.GqlQuery("SELECT * FROM MetaData WHERE path = :1", body)

jsonObject = {}

for metadata in query:
  jsonObject[metadata.keystr] = metadata.value
  
self.response.headers['Content-Type'] = 'text/plain'
self.response.out.write(simplejson.dumps(jsonObject)) 
\end{code}
\caption{HTTP Response.\label{serverimpl-json-response}}
\end{center}
\end{figure*}

\subsubsection{Programmer's Manual}
For more information on how the advert server is implmented, we suggest reading
the \emph{Programmer's Manual} (Appendix \ref{progman}).

\subsection{Authentication and Privacy}
As described in Section \ref{serverdesign-auth} we will implement two servers,
one without user authentication (i.e. a public server without any guarantees),
and a server which requires authentication through Google Accounts. Below we
will show how an authenticated server is implemented.

\subsubsection{Google Accounts}
The most basic and obvious way to achieve authentication is single user
authentication (through Google Accounts). Basically only the owner is allowed to
use the advert service. According to the documentation provided by Google, there
is no variable that indicates if a user is owner of the application. Nonetheless
there is another function called \texttt{is\_current\_user\_admin()}, which is
available in the \texttt{google.appengine.api.users} package. By default the
owner of the application is administrator, and additional administrators can be
added through the administration panel. This fits our needs perfectly.

Once a client logged in through the Google login procedure (e.g. a Google login
page), the request handler at the server returns a user object, as shown in
Figure \ref{serverimpl-auth}. This object can then be used to identify a
user.

\begin{figure*}[ht] %[placement] where placement is h,t,b,p
\begin{center}
\begin{code}
user   = users.get_current_user()

if not user:
    self.error(403)
    self.response.headers['Content-Type'] = 'text/plain'
    self.response.out.write('Not Authenticated')
    return

if not users.is_current_user_admin():
    self.error(403)
    self.response.headers['Content-Type'] = 'text/plain'
    self.response.out.write('No Administrator')
    return
    
...
\end{code}
\caption{Authenticating a User.\label{serverimpl-auth}}
\end{center}
\end{figure*}

Other functions included in the \texttt{google.appengine.api.users} package are:

\begin{itemize} 
\item \texttt{create\_login\_url(dest\_url)}; which returns a URL that, when
visited, will prompt the user to sign in using a Google account, then redirect
the user back to the URL given as \texttt{dest\_url}.
\item \texttt{create\_logout\_url(dest\_url)}; which returns a URL that, when
visited, will sign the user out, then redirect the user back to the URL given as
\texttt{dest\_url}.
\end{itemize}

% \subsubsection{Data Matching}
% For our \texttt{find()} function, we will need some internal `match' function,
% which can query the database to see if the specified data item is present in the
% datastore. For this matching, we will have an internal function which is able to
% match data received from the client with data present in the datastore. 
% %TODO: <more to follow>


