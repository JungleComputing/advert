\section{Conclusion}
\label{conclusion}
In this thesis we have tried to use the Google App Engine, which is a platform
for developing and hosting web applications, for scientific purposes. We
attempted to build an application storage server (also known as advert server)
for Ibis. We designed such a server and implemented it in the Python programming
language, which is the main programming language of the Google App Engine. In
addition we designed a transfer protocol over HTTP, making use of JSON
\cite{json-www}, and a client library which we implemented in Java.

The client library creates a transparent interface for the user, implementing
all server functionality. The client library does the authentication, parameter
encoding, sending/receiving over HTTP, and error handling for the user. This
library can now be used by other applications to store application data at the
Google App Engine.

We also used our client library in two Ibis applications; the JavaGAT
\cite{javagat-www}, and the IPL \cite{ipl-www}. We made a Google App Engine
AdvertService Adaptor for JavaGAT. The advantage of our solution is that it is
much more scalable, compared to the generic AdvertService Adaptor, due to
Google's distributed datastore and replication of the web application.

The second application of our advert server is a bootstrap mechanism for the
IPL server. So far, no such mechanism existed. Our IPL server bootstrap
mechanism stores locations of IPL servers at a fixed address (i.e. the address
of our advert server), after which clients can easily discover the IPL server
by looking up an IPL server address at our advert server. Our addition to the
IPL server and applications solves configuration problems, since a user can now
hard code the advert server address in his/her Ibis application.

We evaluated our solution by benchmarking various parts of the system. From our
benchmarks we can conclude that Google has various limitations with respect to
bandwidth, response time, and number of bytes sent in one call. Also the
completion time of various functions of the system increases as the size of
objects to be sent/received increases, because of the internet latency. In
addition, the completion time increases if more clients connect in parallel,
because Google needs to replicate our server in order to process all requests in
parallel, which takes extra time. Finally we noticed that the completion time
does not increase if the number of items stored in the datastore increases
(i.e. more items need to be searched in order to return the correct item).

\subsection{Future Work}
\label{conclusion-future}