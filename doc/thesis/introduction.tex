\section{Introduction}
\label{introduction}
In April 2008 \cite{app-engine-intro}, Google launched a platform for building
and hosting web applications on their infrastructure, called the \emph{Google App
Engine} \cite{app-engine-www}. Based on cloud computing technology, the Google
App Engine uses multiple servers to run an application, written in the
\emph{Python} programming language \cite{python-www}, and store data. In
addition, Google automatically adjusts the number of servers to handle requests
simultaneously. All is offered for free by Google, provided that there are
certain quotas (e.g. bandwidth, disk space, etc.). In addition, one could choose
to sign up for a billable account, where one is billed after the quota is
exceeded.

Our research question is: \emph{``To what extent can we use the Google App Engine
for scientific purposes, with respect to Ibis?''}. Obviously, Google did not
intend the App Engine to be used for such purposes, since the Google App Engine
runs our applications in a sandbox (i.e. a secure environment that provides
limited access to the underlying operating system) \cite{app-engine-sandbox}. One
of the most usable parts of the App Engine is the distributed database, called
the \emph{datastore}. The datatsore is a schemaless object datastore, with a
query engine and atomic transactions. The Python interface includes a rich data
modeling API and a SQL-like query language called GQL
\cite{app-engine-datastore}. The distributed datastore, could serve well as
\emph{application storage service} (also known as \emph{advert server} -- from
now on we will use these two terms interchangeably) for \emph{Ibis}. As far as we
know, the Google App Engine has not been used as an application storage server.

To give an example, imagine us to have some sort of computational workflow, where
the output of each unit is the input of the next unit (see Figure
\ref{img-workflow}). Now suppose we want to store the output of the first unit,
in our example the Gaussian Blur, in order for another to find it, and continue
processing. To achieve this we will set up a central application storage service
(called the advert server), at which we can store intermediate workflow output,
in order for other units (e.g. the Photocopy FX) to find and process it again.
This is just a basic example of what an application storage service could be used
for. Real-world examples might be more complex.

\begin{figure*}[ht] %[placement] where placement is h,t,b,p
\begin{center}
\includegraphics[width=14cm]{./figures/image_workflow.pdf} 
\caption{An Example of a Computational Workflow.\label{img-workflow}}
\end{center}
\end{figure*}

As an example for our application storage service, we took a closer look at the
\emph{JavaGAT AdvertService} \cite{javagat-www}. This provided us with main
functionality for our own advert server, for adding, getting, deleting, and
finding objects in the Google datastore. Secondly we wrote a client library in
the Java programming language, which communicates over HTTP with the advert
server, running at Google. Other Java programs can now use the client library to
store and retrieve (binary) application data at the Google App Engine.

In addition to this server, we wrote a Java client to communicate to the server
over HTTP(S), as can be seen from Figure \ref{introduction-overview}. Once this
library is imported into Ibis, Ibis applications can use the library to store
their data items in the data store.

\begin{figure*}[ht] %[placement] where placement is h,t,b,p
\begin{center}
\includegraphics[width=14cm]{./figures/project_design.pdf} 
\caption{An Overview of our Project.\label{introduction-overview}}
\end{center}
\end{figure*}

As a result from building our advert server similar to JavaGAT's AdvertService,
we could easily implement a JavaGAT AdvertService \emph{adaptor} for our Google
App Engine advert server. Also we implemented a \emph{IPL server bootstrap
mechanism} for Ibis \cite{ipl-www}, where IPL servers can register themselves in
order to be found by other Ibis applications.

The rest of this paper is organized as follows. In Section \ref{related} we look
at related work, concerning Ibis and the Google App Engine. In Sections
\ref{serverdesign} and \ref{serverimpl} we describe our advert server design and
implementation, respectively, followed by the implementation of our Advert client
library in Section \ref{clientimpl}. Next we look at two applications of Ibis in
Section \ref{applications}. Finally we present our evaluation in Section
\ref{evaluation}, followed by our conclusion and future work (Section
\ref{conclusion}).
