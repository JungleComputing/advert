\section{Introduction}
\label{introduction}
In April 2008 \cite{app-engine-intro}, Google launched a platform for building
and hosting web applications on their infrastructure, called the \emph{Google App
Engine} \cite{app-engine-www}. Based on cloud computing technology, Google App
Engine uses multiple servers to run an application and store data. In addition,
Google automatically adjusts the number of servers to handle requests
simultaneously. All is offered for free by Google, provided that there are
certain quotas (e.g. bandwidth, disk space, etc.). In addition, one could
choose to sign up for a billable account, where one is billed after the quota
is exceeded.

As of yet, the Google App Engine is not used for scientific purposes. Therefore
our research question is: \emph{``To what extent can we use the Google App Engine
for scientific purposes?''}. Since the Google App Engine runs our applications in
a sandbox (i.e. a secure environment that provides limited access to the
underlying operating system) \cite{app-engine-sandbox}, one of the most usable
parts of the App Engine is the distributed database, called the \emph{datastore}.
The datatsore is a schemaless object datastore, with a query engine and atomic
transactions. The Python interface includes a rich data modeling API and a
SQL-like query language called GQL \cite{app-engine-datastore}.

The Google App Engine Datastore could well serve as an \emph{Application
Storage service}. 
