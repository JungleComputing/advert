\section{Client Library}
\label{clientimpl}
Now that we have designed and implemented our application storage server, it is
time to build a client library, which communicates to the application storage
server, in order to create transparency for the user. Our client library is
completely written in Java and needs Java version 1.5 to compile.

\subsection{Public Functions}
Similar to the public functions specified in the server design (Section
\ref{serverdesign-pub}), we specified our client library functions. These
functions are also very similar to the JavaGAT API, which we will explain
further in Section \ref{applications-advertservice}. The following public
functions are available:

\begin{itemize}
	\item \texttt{void add(MetaData, Object)} 
	\item \texttt{void delete(path)}
	\item \texttt{String[] find(MetaData)}
	\item \texttt{Object getInstance(path)}
	\item \texttt{MetaData getMetaData(path)}
	\item \texttt{String getPWD()}
	\item \texttt{void setPWD(path)} 
\end{itemize}

All functions do exactly as described in Section \ref{serverdesign-pub}. In
addition to the server's public functions, two functions have been added to the
Advert library, being the \texttt{getPWD()} and the \texttt{setPWD()}. This is
implemented if a user wishes to use relative paths in stead of absolute paths.

In addition, we also added two constructors. One is the default constructor,
which creates an empty \texttt{MetaData} object. The other is a constructor
which takes a serialized form as an argument and returns a \texttt{MetaData}
object with the serialized key-value pairs already inserted. The serialized
form needs to look like this: \texttt{key1=value1,key2=value2,key3=value3}.

\subsection{Data Types}
In addition to public functions, we also implemented an extra data type, called
\texttt{MetaData}. In essence this is for storing key-value pairs, just like
meta data is stored at the server. The only difference is that a
\texttt{MetaData} object has more functionality than just key-value pairs.

\begin{itemize}
	\item \texttt{String get(key)}; Gets the value associated to the provided key 
	\item \texttt{String getData(int)}; Gets the value associated to the key
		retrieved by \texttt{getKey(int)}
	\item \texttt{String getKey(int)}; Gets the \texttt{i}-th key of the
		\texttt{MetaData}
	\item \texttt{boolean match(MetaData)}; Match two \texttt{MetaData} objects 
	\item \texttt{void put(key, value)}; Put an entry in the \texttt{MetaData} object 
	\item \texttt{String remove(key)}; Removes an entry specified by the provided
		key
	\item \texttt{int size()}; Returns the number of entries in the \texttt{MetaData} 
\end{itemize}

As one can see, a significant number of user functions is added to the
\texttt{MetaData} object. A lot of functions adhere to the \texttt{MetaData}
object provided by the JavaGAT \emph{AdvertService}. We will look deeper into
this when we discuss the AdvertService in Section
\ref{applications-advertservice}.

\subsection{Transfer Protocol}
Below we will describe the transfer protocol in more detail, as from a client
point of view. All our communication is done by our \texttt{Communication}
class, which is called by the public functions stated above. 

First we will describe some basics of how Java manages HTTP connections. After
that, we will take a look at the actual protocol, which is stated in both the
server design and implementation.

\subsubsection{HTTP(S) Connections}
Since our client is implemented in pure Java, it is useful to make use of the
\texttt{java.net} class, which allows us to make HTTP connections to the
App Engine. By creating an URL object like shown in Figure \ref{clientimpl-url}.

\begin{figure*}[ht] %[placement] where placement is h,t,b,p
\begin{center}
\begin{code}
URL url = new URL("http://ibis-advert.appspot.com/");
HttpURLConnection httpc = (HttpURLConnection) url.openConnection();
\end{code}
\caption{Opening an HTTP Connection.\label{clientimpl-url}}
\end{center}
\end{figure*}

Now it is possible with the use of the various functions in both classes to get
all information needed; like status-codes, HTTP Headers, and the message body.
This we will discuss below when we talk about the HTTP response.

In addition to HTTP connections, it is also possible in Java to make HTTPS
connections. This is done automatically by the \texttt{HttpURLConnection}
class, when a URL with scheme \texttt{https} is opened.

% however, this is not as straightforward as making HTTP connections.
% To make an HTTPS connection we will need to import the
% \texttt{java.security.Security} class (amongst others). In addition, if we look
% at Figure \ref{clientimpl-https}, we see that the code adds us to the
% \texttt{security.provider} list in \texttt{java.security}. After these lines of
% code we will be able to make an HTTPS connection just like we did with making
% HTTP connection above (i.e. \texttt{url.openConnection()}). All sample code is
% fully operational at our branch in the JavaGAT SVN Repository .
% 
% \begin{figure*}[ht] %[placement] where placement is h,t,b,p
% \begin{center}
% \begin{code}
% Security.addProvider(new com.sun.net.ssl.internal.ssl.Provider());
% 
% Properties properties = System.getProperties();
% 
% String handlers = System.getProperty("java.protocol.handler.pkgs");
% if (handlers == null) { //nothing specified yet (expected case)
%     properties.put("java.protocol.handler.pkgs",
%         "com.sun.net.ssl.internal.www.protocol");
% } 
% else { //something already there, put ourselves out front
%     properties.put("java.protocol.handler.pkgs",
%         "com.sun.net.ssl.internal.www.protocol|".concat(handlers));
% }
% System.setProperties(properties);
% \end{code}
% \caption{Setting up an HTTPS Connection.\label{clientimpl-https}}
% \end{center}
% \end{figure*}

\subsubsection{HTTP POST Method}
Secondly it is of great importance to create an HTTP POST request for both
authentication and sending data to the App Engine. The simplest option is to send
a various number of URL encoded strings. An example of making such an HTTP POST
request can be seen in Figure \ref{clientimpl-post}. We enable sending output to
the connection just created, calling \texttt{urlc.setDoOutput(true);}. After
making this call we are able to open an OutputStreamWriter to which we can write
our POST data. Multiple fields are separated by an ampersand, and variable names
and values are separated by the equal sign. In this example we write the author's
name and content to an HTTP POST request.

\begin{figure*}[ht] %[placement] where placement is h,t,b,p
\begin{center}
\begin{code}
/* Setting up POST environment. */
httpc.setRequestMethod("POST");
httpc.setDoOutput(true);
OutputStreamWriter out = new OutputStreamWriter(httpc.getOutputStream());

/* Writing POST data. */
out.write("author=bbn230&content=test");
out.close();
\end{code}
\caption{Making an HTTP POST request.\label{clientimpl-post}}
\end{center}
\end{figure*}

% \paragraph{Sending Binary Data}
% In addition to just making HTTP post requests, it is also possible to send binary
% data (commonly used for uploading files to web servers via HTTP). Again, this is
% a step more difficult. We have to make sure the server expects a binary string of
% data. This is shown in Figure \ref{clientimpl-binpost}. After the request has
% been made, the content can be extracted from the message body by the server and
% can be stored accordingly.
% 
% \begin{figure*}[ht] %[placement] where placement is h,t,b,p
% \begin{center}
% \begin{code}
% /* Setting up POST environment. */
% urlc.setDoOutput(true);
% urlc.setRequestProperty("Content-Type", "application/octet-stream");
% \end{code}
% \caption{Making a binary HTTP POST request.\label{clientimpl-binpost}}
% \end{center}
% \end{figure*}

\subsubsection{HTTP Response}
After the HTTP request is sent, we are able to receive the HTTP response. To
begin with, we will download the response headers sent by the web server, after
which we can retrieve the body.

\paragraph{Headers}
How to recieve the headers of an HTTP response is shown in Figure
\ref{clientimpl-headers}. Already, by receiving the response code, we can
distinguish errors from successful requests. Response codes starting with a 1 are
informational and those starting with a 2xx mean success. If the response code
is starting with a 3xx are used for redirection, and 4xx and 5xx are used for
errors (client and server respectively). For example: 200 stands for success, 403 for
authentication required, and 404 for not found.

\begin{figure*}[ht] %[placement] where placement is h,t,b,p
\begin{center}
\begin{code}
/* Retrieving headers. */
System.out.println(httpc.getResponseCode());
System.out.println(httpc.getRequestMethod());
for (int i = 0; httpc.getHeaderField(i) != null; i++) {
    System.out.println(httpc.getHeaderField(i));
}
\end{code}
\caption{Retrieving HTTP response headers.\label{clientimpl-headers}}
\end{center}
\end{figure*}

After checking the response code, we can retrieve additional headers. In our case
(using the App Engine as our web server), we have listed an example of receiving
a succesful response (Figure \ref{clientimpl-200}). Again, a status header is
sent, followed by the server type, the date, and some other info. Note that
``Content-Length'' can be really useful to determine our buffer size in the next
step.

\begin{figure*}[ht] %[placement] where placement is h,t,b,p
\begin{center}
\begin{code}
HTTP/1.0 200 OK
Server: Development/1.0
Date: Wed, 11 Mar 2009 11:25:05 GMT
Cache-Control: no-cache
Content-Type: image/gif
Content-Length: 18006
\end{code}
\caption{Example of HTTP response headers.\label{clientimpl-200}}
\end{center}
\end{figure*}

\paragraph{Cookies}
\label{clientimpl-cookies}
A `special' type of HTTP header that can be retrieved from sending an HTTP
request are cookies. Just like all the other headers, it is retrieved with the
code stated above. When printed to screen, it will look similar to the request
shown in Figure \ref{clientimpl-cookies-resp}. This code above is taken from
retrieving the URL \url{http://www.google.nl/}. As you can see, this cookie
consists out of multiple parts, but would normally be stored as one cookie in your
browser. In this case, the cookies name is ``PREF'', and its content is
``ID=8f3c26(\ldots)zrigB6''. Finally it comes with an expiration time and date,
a path, and a domain. This information will be needed when we want to authenticate
ourselves to the App Engine.

\begin{figure*}[ht] %[placement] where placement is h,t,b,p
\begin{center}
\begin{code}
HTTP/1.1 200 OK
Cache-Control: private, max-age=0
Date: Wed, 11 Mar 2009 11:57:38 GMT
Expires: -1
Content-Type: text/html; charset=ISO-8859-1
Set-Cookie: 
  PREF=ID=8f3c262a9b38330e:TM=1236772658:LM=1236772658:S=L2ORt13rUlzrigB6; 
  expires=Fri, 11-Mar-2011 11:57:38 GMT; path=/; domain=.google.nl
Server: gws
\end{code}
\caption{An HTTP response including cookies.\label{clientimpl-cookies-resp}}
\end{center}
\end{figure*}

\paragraph{Message Body}
Once we've received all headers and parsed them accordingly, we are able to get
the actual body. As we can see from Figure \ref{clientimpl-body}, the most
important thing is getting the stream, provided by the \texttt{getInputStream()}
function. After we have done this, we can basically do anything with our
InputStream. In this case we print it to the console, but we could also save it
in a buffer and manipulate it, or write it do disk, etc. After we are done
receiving the InputStream, we close the InputStream, effectively closing the
connection.

\begin{figure*}[ht] %[placement] where placement is h,t,b,p
\begin{center}
\begin{code}
BufferedReader in = 
  new BufferedReader(new InputStreamReader(httpc.getInputStream()));
String inputLine;

while ((inputLine = in.readLine()) != null) {
    System.out.println(inputLine);
}
in.close();
\end{code}
\caption{Retrieving the HTTP response's body.\label{clientimpl-body}}
\end{center}
\end{figure*}

\subsubsection{Object Encoding}
\label{clientimpl-encoding}
As shown in our server design (Section \ref{serverdesign-encoding}), we are using
JSON for encoding our objects before sending them to the server. In our client
library, we make use of the \emph{Json-lib} library \cite{json-lib-www}, which
fully implements the JSON format. A simple examples is given in Figure
\ref{clientimpl-json}, where we transform a \texttt{MetaData} object into a
serialized JSON object.

\begin{figure*}[ht] %[placement] where placement is h,t,b,p
\begin{center}
\begin{code}
JSONObject    jsonobj = new JSONObject();
Iterator<String> itr  = metaData.getAllKeys().iterator();

while (itr.hasNext()) {
	String key   = itr.next();
	String value = metaData.get(key);
	jsonobj.put(key, value);
}

String serialForm = jsonobj.toString();
\end{code}
\caption{JSON Encoding in Java.\label{clientimpl-json}}
\end{center}
\end{figure*}

This serialized form can now be send to the server using the HTTP POST method,
as described above.

\subsection{Authentication and Privacy}
\label{clientimpl-auth}
Finally we will describe how authentication is done. All functionality which
concerns authentication is done in the \texttt{Authentication} class, present
in the Advert client library. Logging in is done through Google Accounts (as
can be seen from Sections \ref{serverdesign} and \ref{serverimpl}).

\subsubsection{Google Login Page}
We have looked at the login page of a typical App Engine application, and we
have stripped the unnecessary HTML from the login form. As we can see from the
result shown in Figure \ref{clientimpl-loginform}, the login form is quite
comprehensive. The login form has a lot of extra (hidden and redundant) fields,
which we did not expect to find.

\begin{figure*}[ht] %[placement] where placement is h,t,b,p
\begin{center}
\begin{code}
<form id="gaia_loginform" 
action="https://www.google.com/accounts/ServiceLoginAuth?service=ah
    &amp;sig=d71ef8b8d6150b23958ad03b3bf546b7" 
  method="post"
  onsubmit="return(gaia_onLoginSubmit());">

<input type="hidden" name="ltmpl" value="gm">
<input type="hidden" name="continue" id="continue"
  value="http:// bbn230.appspot.com/_ah/login?
  continue=http://bbn230.appspot.com/" />
<input type="hidden" name="service" id="service" value="ah" />
<input type="hidden" name="ltmpl" id="ltmpl" value="gm" />
<input type="hidden" name="ltmpl" id="ltmpl" value="gm" />
<input type="hidden" name="ahname" id="ahname" value="Personal" />
<input type="text" name="Email"  id="Email" size="18" value="" />
<input type="password" name="Passwd" id="Passwd" size="18" />
<input type="checkbox" name="PersistentCookie" id="PersistentCookie" 
  value="yes" />
<input type="hidden" name='rmShown' value="1" />
<input type="submit" class="gaia le button" name="signIn" value="Sign in" />
<input type="hidden" name="asts" id="asts" value="">
</form>
\end{code}
\caption{Stripped down version of the App Engine
login page.\label{clientimpl-loginform}}
\end{center}
\end{figure*}

We are not sure what all hidden fields do, and if they are really necessary, but
simulating this login page by posting all this would make our HTTP POST request
quite large.

\subsubsection{Google Account Authentication API}
An alternative for simulating the login page is the \emph{Account
Authentication API} \cite{account-auth-api}. The Account Authentication API
comes in two forms. One form for installed \emph{Google Apps}
\cite{google-apps-www}, which is called \emph{ClientLogin}, the other is for Web
Apps, and is called \emph{OAuth} or \emph{AuthSub}.

\paragraph{ClientLogin}
Typically, ClientLogin is used for Installed applications that need to access
Google services protected by a user's Google or Google Apps. To make use of this
API, we will make a HTTP POST request to
\texttt{https://www.google.com/accounts/ClientLogin}, to which we send our
credentials as shown in Figure \ref{clientimpl-clientlogin}.

\begin{figure*}[ht] %[placement] where placement is h,t,b,p
\begin{center}
\begin{code}
StringBuilder content = new StringBuilder();
content.append("Email=").append(URLEncoder.encode("johndoe@gmail.com", 
  "UTF-8"));
content.append("&Passwd=").append(URLEncoder.encode("north23AZ", "UTF-8"));
content.append("&service=").append(URLEncoder.encode("ah", "UTF-8"));
content.append("&source=").append(URLEncoder.encode("Google App Engine", 
  "UTF-8"));
\end{code}
\caption{Preparing our ClientLogin credentials.\label{clientimpl-clientlogin}}
\end{center}
\end{figure*}

We created a Google Account for testing purposes and made a successful connection
using that account. The response of the ClientLogin is shown in Figure
\ref{clientimpl-clientlogin-response}.

\begin{figure*}[ht] %[placement] where placement is h,t,b,p
\begin{center}
\begin{code}
HTTP/1.1 200 OK
Content-Type: text/plain
Cache-control: no-cache, no-store
Pragma: no-cache
Expires: Mon, 01-Jan-1990 00:00:00 GMT
Date: Mon, 16 Mar 2009 14:07:25 GMT
X-Content-Type-Options: nosniff
Content-Length: 497
Server: GFE/1.3
----
SID=DQA(...)bjG
LSID=DQA(...)nSV
Auth=DQA(...)ub4
\end{code}
\caption{A response from ClientLogin.\label{clientimpl-clientlogin-response}}
\end{center}
\end{figure*}

Currently, `SID' and `LSID' are not used by the \emph{Google API}, so we just
need to extract the `Auth' value. Usually this value can be used directly on a
service its API (when providing a developer key), but unfortunately, the Google
App Engine is not listed in the Google data API library \cite{service-api}.
There is a workaround for still being able to connect to the App Engine, by
connecting to a App Engine's login page like Figure \ref{clientimpl-aelogin}.
In this example, \texttt{Auth} is a variable that contains the token acquired
by the example of Figure \ref{clientimpl-clientlogin-response}. 

\begin{figure*}[ht] %[placement] where placement is h,t,b,p
\begin{center}
\begin{code}
url = new URL("http://bbn230.appspot.com/_ah/login?auth="+ Auth);
\end{code}
\caption{Logging In for a Session Cookie.\label{clientimpl-aelogin}}
\end{center}
\end{figure*}

Connecting to this page will return a cookie with a session ID, just like the
one described in Section \ref{clientimpl-cookies}, only with a token called
`ACSID', with which we can identify ourselves to the App Engine for every
subsequent request.

\paragraph{AuthSub}
Web applications that need to access services protected by a user's Google or
Google Apps (hosted) account can do so using the Authentication Proxy service. To
maintain a high level of security, the proxy interface, called AuthSub, enables
the web application to get access without ever handling their users' account
login information. Before using, verify that the Google service to be accessed
supports the Authentication service. Some Google services may allow access only
to web applications that are registered and use secure tokens. We have not tried
AuthSub to authenticate ourselves at the App Engine. If the ClientLogin remains
to fail, we could consider AuthSub as our backup. We will not use it as our
primary authentication mechanism because if our users would want to install an
advert server, it would require extra effort, because they need to register
themselves first before this service can be used.

\subsubsection{Persistent Authentication}
Also, to maintain session with the server, we want to pass a session ID through
cookies. Also, this can be done through Java by setting the appropriate headers.
For example, if we want to send a cookie with the name ``SID'' and the content
``abcde'', we could code this in Java as shown in Figure
\ref{clientimpl-cookies-req}. This code will add the cookie to the HTTP request,
after which the response headers and body can be fetched.

\begin{figure*}[ht] %[placement] where placement is h,t,b,p
\begin{center}
\begin{code}
urlc.addRequestProperty("Cookie", "SID=abcde");
\end{code}
\caption{Sending cookies in Java.\label{clientimpl-cookies-req}}
\end{center}
\end{figure*}

Because the authentication cookies, as provided by the Google App Engine,
generally expire after 24\,hours. This means that, unless the user restarts the
Advert library, the user will receive an \emph{HTTP 403 Authentication} error,
once an Advert server runs over 24\,hours consequently. 

To prevent the user from resetting the Advert server himself, we added a
\emph{Persistent Authentication} class. This class is actually a daemon thread,
which is created once the server has been authenticated. This server thread
parses the expiry date, which comes with the authentication cookie, and goes to
sleep 0.9 times the calculated expiry interval. As soon as the cookie is about
to expire, the daemon thread wakes up, and does a \texttt{NOOP} (NO OPeration)
call (i.e. a HTTP GET request of the main page), after which the App Engine
will automatically send a new authentication cookie and a new expiry date along
with it. Accordingly, we calculate the new expiry interval and go to sleep
again.

If the Advert service is used on a frequent basis, it could occur that by using
one of the Advert server's functions, a new (i.e. refreshed) cookie is sent
alongside the server response. If this is the case, the library will
automatically update the cookie present in the Persistent Authentication class,
so the daemon thread won't have to perform unnecessary \texttt{NOOP} calls.

The Persistent Authentication class is the only class where the authentication
cookie is stored. If another function wants to reference or update the
authentication cookie, \texttt{getCookie()} or \texttt{setCookie()} functions
can be used, respectively.
