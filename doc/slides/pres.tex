% You should give sufficient motivation to the problem you are dealing with,
% present the background, the options you had in attacking the problem, and the
% reasons you are following some specific policies in your research. This is a
% non-exhaustive list. Depending on the project, you can think of various
% additional aspects to present (e.g., what results would be considered good and
% what bad, etc.).

\documentclass{beamer}

\usepackage{beamerthemesplit}
\usepackage{fancyvrb}
\usepackage{multicol}

\title{A Cloud-based Application Storage Service \newline for Ibis}
\author{Bas Boterman}
\date{\today}

\begin{document}

\frame
{
	\titlepage
	\begin{figure}[h]
	\begin{center}
	\includegraphics[width=2cm]{msc_logo.png} 
	\end{center}
	\end{figure}
}

\section[Outline]{}
\frame
{
	\tableofcontents
}

\section{Introduction}
%\subsection{Background}
\frame
{
	\frametitle{Background}
	\begin{itemize}
    	\item <1->April 7, 2008, the Google App Engine was introduced
    	\item <2->In what extent can we use the Google App Engine for scientific
      		purposes?
      	\item <3->Robust distributed datastore with transactions and queries
    	\item <4->This project is about designing an application storage server
    		for Ibis
    \end{itemize}
}


%\subsection{Thesis}
\frame[t]
{
	\frametitle{Thesis}
	A Cloud-based Application Storage Service for Ibis 
}

\frame[t]
{
	\frametitle{Thesis}
	A \textbf{Cloud-based} Application Storage Service for Ibis
	\newline
	\uncover<2->{
		\begin{quote}
			A style of computing in which dynamically scalable resources are provided as
			a service over the Internet.
		\end{quote}
	
		E.g. the Google App Engine.
	}
}

\frame[t]
{
\frametitle{Thesis}
	A Cloud-based \textbf{Application Storage Service} for Ibis
	\newline
	\uncover<2->{
		\begin{quote}
			A service used by applications to store and retrieve application data.
		\end{quote}
		
		I.e. an Advert service.
	}
}

\frame[t]
{
\frametitle{Thesis}
	A Cloud-based Application Storage Service for \textbf{Ibis}
	\newline
	\uncover<2->{
		\begin{quote}
			The main goal of the Ibis project is to create an efficient Java-based
			platform for grid computing.
		\end{quote}
	}
	\begin{itemize}
		\item<2-> JavaGAT (Grid Application Toolkit)
		\item<2-> IPL (Ibis Portability Layer)
	\end{itemize}
}

%\subsection{Google App Engine}
\frame
{
	\frametitle{Google App Engine}
	\begin{quote}
		A platform for building and hosting web applications on
		the Google infrastructure for free.
	\end{quote}
	
	\begin{itemize}
		\item Python programming language
		\item Web application framework
		\item Authentication through Google Accounts
		\item 10GB traffic per day
		\item 1GB of data storage
	\end{itemize}
}

%\subsection{Properties}
\frame
{
	\frametitle{(Dis)Advantages}
	\uncover<1->{Advantages:}
	\begin{itemize}
		\item <1->Powerful (distributed) database with query engine and
			transactions
		\item <1->Replicating servers to improve scalability 
		\item <1->Authentication through Google Accounts
		\item <1->Built-in frameworks (Django, WebOb, PyYAML)
	\end{itemize}

	\uncover<2->{Disadvantages:}
	\begin{itemize}
		\item <2->Python 2.5 (limited functionality, compared to 3.0)
		\item <2->HTTP(S) only
		\item <2->Sandbox (no \texttt{fork()}, no threads, no file system)
	\end{itemize}
}

\section{Project Details}
%\subsection{Design}
\frame
{
	\frametitle{Project Overview}
	\begin{figure}
	\begin{center}
	\includegraphics[width=10cm]{proj_design.pdf} 
	\caption{Schematic Overview of the Project}
	\end{center}
	\end{figure}
}

\frame
{
	\frametitle{Server Design}
	Public functions:
	\begin{itemize}
		\item <1->\texttt{add(data, metadata, path)}
		\item <1->\texttt{del(path)}
		\item <1->\texttt{get(path)}
		\item <1->\texttt{getmd(path)}
		\item <1->\texttt{find(metadata)}
	\end{itemize}
	\begin{itemize}
		\item <2->\texttt{data}: Text
		\item <2->\texttt{metadata}: Key-value pairs of Strings
		\item <2->\texttt{path}: String
	\end{itemize}
}

\frame
{
	\frametitle{Client Library}
	\begin{itemize}
		\item Written in Java
		\item Implements the GAE Advert server's protocol
		\begin{itemize}
			\item Base64 encoding
			\item JSON encoding
			\item Error handling 
        \end{itemize}
        \item Establishes connections over HTTP(S)
		\item Takes care of authentication
		\begin{itemize}
			\item Authentication through \emph{Google ClientLogin}
			\item Automatically refresh authentication cookie
		\end{itemize} 
	\end{itemize}
}

\section{Benchmarks}
%\subsection{Server Functions}
\frame
{
	\frametitle{Add()}
	\begin{figure}[t]
	\begin{center}
	\includegraphics[trim=4cm 4cm 4cm 5cm, width=8cm]{add_obj.pdf} 
	\caption{Adding Objects of Variable Size}
	\end{center}
	\end{figure}
}

\frame
{
	\frametitle{Add()}
	\begin{figure}[t]
	\begin{center}
	\includegraphics[trim=4cm 4cm 4cm 5cm, width=8cm]{add_md.pdf} 
	\caption{Adding Objects with Variable Meta Data}
	\end{center}
	\end{figure}
}

\frame
{
	\frametitle{Get()}
	\begin{figure}[t]
	\begin{center}
	\includegraphics[trim=4cm 4cm 4cm 5cm, width=8cm]{get_amt.pdf} 
	\caption{Retrieving Objects with Variable Meta Data}
	\end{center}
	\end{figure}
}

\frame
{
	\frametitle{Find()}
	\begin{figure}[t]
	\begin{center}
	\includegraphics[trim=4cm 4cm 4cm 5cm, width=8cm]{find_amt.pdf} 
	\caption{Finding Meta Data}
	\end{center}
	\end{figure}
}

%\subsection{Clients in Parallel}
\frame
{
	\frametitle{Parallel Benchmarks}
	\textbf{TODO:} under construction!
}

\frame
{
	\frametitle{Benchmark Evaluation}
	\begin{itemize}
    	\item Latency increases if object size/meta data size increases
    	\item Latency remains same if number of objects in datastore increase
    	\item Latency increases if multiple clients connect at the same time
    	\item Server times out if too many items are present in datastore (latency
    		$>$ 30 seconds)
    	\item Server temporarily exceeds quota if over 5MB is submitted within
    		60 seconds
    	\item Server fails if over 500 data items are manipulated in one call
    	\item ClientLogin fails if 100 clients try to authenticate in parallel
    \end{itemize}
}

\section{Applications}
%\subsection{JavaGAT}
\frame
{
	\frametitle{JavaGAT}
	\begin{quote}
		JavaGAT offers a set of coordinated, generic and flexible APIs for accessing
		grid services from application codes, portals, data managements systems, etc.
	\end{quote}
	
	\begin{itemize}
		\item File operations (File, LogicalFile, RandomAccessFile, Endpoint)
		\item File stream operations (FileInputStream, FileOutputStream)
		\item Job submission (ResourceBroker)
		\item Monitoring (Monitorable)
		\item Access to information services (AdvertService)
	\end{itemize}
}

\frame[t]
{
	\frametitle{JavaGAT structure}
	\begin{figure}[t]
	\begin{center}
	\includegraphics[width=7cm]{gat-design.png} 
	\caption{Current JavaGAT Design}
	\end{center}
	\end{figure}
}

\frame[t]
{
	\frametitle{JavaGAT structure}
	\begin{figure}[t]
	\begin{center}
	\includegraphics[width=7cm]{gat-mydesign.png} 
	\caption{JavaGAT with App Engine AdvertService Adaptor}
	\end{center}
	\end{figure}
}

\frame
{
	\frametitle{App Engine AdvertService Adaptor}
	\begin{itemize}
		\item As of yet, only a local AdvertService existed
		\item Scalable design (local AdvertService does not scale)
		\item Completely transparent to user
		\item Path handling is done locally
    \end{itemize}
}

%\subsection{IPL}
\frame
{
	\frametitle{IPL}
	\begin{quote}
		The IPL is a communication library is specifically designed for usage in a
		grid environment.
	\end{quote}
	\begin{itemize}
    	\item Communication library for distributed applications
    	\item All applications communicate through an IPL registry server
    	\item IPL registry server address needs to be passed down to each Ibis
    		application
    \end{itemize}
}

\frame[t,containsverbatim]
{
	\frametitle{IPL Example}
	\begin{Verbatim}[fontsize=\small,gobble=2]
		$ $IPL_HOME/bin/ipl-server --events

		Ibis server running on 130.37.20.18-8888~bbn230
		List of Services:
		    Bootstrap service on virtual port 303
		    Central Registry service on virtual port 302
		    Management service
		Known hubs now: 130.37.20.18-8888~bbn230
		

		$ $IPL_HOME/bin/ipl-run \
		-Dibis.server.address=130.37.20.18-8888~bbn230 \
		-Dibis.pool.name=test \
		ibis.ipl.examples.Hello
    \end{Verbatim}
}

\frame[t,containsverbatim]
{
	\frametitle{IPL Example Using Advert Server (Server)}
	\begin{Verbatim}[fontsize=\small,gobble=2]
		$ $IPL_HOME/bin/ipl-server --events \
		--advert google://jondoe.appspot.com/identifier \
		--user jondoe@gmail.com --pass north23AZ \
		--metadata author=jondoe,created=24jun,pool=test,color=purple

		Ibis server running on 130.37.20.18-8888~bbn230
		List of Services:
		    Bootstrap service on virtual port 303
		    Central Registry service on virtual port 302
		    Management service
		Known hubs now: 130.37.20.18-8888~bbn230
	\end{Verbatim}
}

\frame[t,containsverbatim]
{
	\frametitle{IPL Example Using Advert Server (Application)}
	\begin{Verbatim}[fontsize=\small,gobble=2]
		$ $IPL_HOME/bin/ipl-run \
		-Dibis.advert.address=google://jondoe.appspot.com \
		-Dibis.advert.username=jondoe \
		-Dibis.advert.password=north23AZ \
		-Dibis.advert.metadata=created=24jun,color=purple \
		-Dibis.pool.name=test \
		ibis.ipl.examples.Hello
    \end{Verbatim}
}

\frame
{
	\frametitle{IPL Registry Bootstrap Server}
	\begin{itemize}
    	\item No server for bootstrapping IPL registries existed
    	\item Large range of IPL registries can be started, whilst only remembering
    		one (advert) address
		\item Different groups of Ibis can be started without (manually) sharing the
			server address
		\item Advert address can be hard-coded in Ibis application 
    \end{itemize}
}

\section{Conclusion}
%\subsection{Verdict}
\frame{
	\frametitle{Conclusion}
	Achievements:
	\begin{itemize}
      \item <1->A Google App Engine Advert server, written in Python
      \item <1->An Advert client library, written in Java
      \item <1->An App Engine AdvertService adaptor for JavaGAT
      \item <1->A registry bootstrap server for IPL
    \end{itemize}
	
	\uncover<2->{Properties:}
	\begin{itemize}
      \item <2->Scalable application and datastore
      \item <2->High bandwidth (until quota reached)
      \item <2->No guarantees
    \end{itemize}
    
    \uncover<3->{Future work:}
    \begin{itemize}
      \item <3->An App Engine Advert server, written in Java
    \end{itemize}
}

%\subsection{Questions}
\frame{
	\frametitle{Questions}
	\begin{center}
		\huge{Any questions?}
	\end{center}
}
\end{document}