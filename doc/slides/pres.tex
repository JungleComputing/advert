\documentclass{beamer}

\usepackage{beamerthemesplit}

\setbeamerfont{code}{size=\tiny}

\title{A Cloud-based Application Storage Service \newline for Ibis}
\author{Bas Boterman}
\date{\today}

\begin{document}

\frame
{
	\titlepage
	\begin{figure}[h]
	\begin{center}
	\includegraphics[width=3cm]{msc_logo.png} 
	\end{center}
	\end{figure}
}

\section[Outline]{}
\frame
{
	\tableofcontents
}

\section{Introduction}
\subsection{Thesis}
\frame[t]
{
	\frametitle{Thesis}
	A Cloud-based Application Storage Service for Ibis 
}

\frame[t]
{
	\frametitle{Thesis}
	A \textbf{Cloud-based} Application Storage Service for Ibis
	\newline
	\uncover<2->{
		\begin{quote}
			A style of computing in which dynamically scalable resources are provided as
			a service over the Internet.
		\end{quote}
	
		E.g. the Google App Engine.
	}
}

\frame[t]
{
\frametitle{Thesis}
	A Cloud-based \textbf{Application Storage Service} for Ibis
	\newline
	\uncover<2->{
		\begin{quote}
			A service used by applications to store and retrieve application data.
		\end{quote}
		
		I.e. an Advert service.
	}
}

\frame[t]
{
\frametitle{Thesis}
	A Cloud-based Application Storage Service for \textbf{Ibis}
	\newline
	\uncover<2->{
		\begin{quote}
			The main goal of the Ibis project is to create an efficient Java-based
			platform for grid computing.
		\end{quote}
	}
	\begin{itemize}
		\item<2-> JavaGAT (Grid Applicatin Toolkit)
		\item<2-> IPL (Ibis Portible Layer)
	\end{itemize}
}

\subsection{Google App Engine}
\frame
{
	\frametitle{Google App Engine}
	\begin{quote}
		A platform for building and hosting web applications on
		the Google infrastructure for free.
	\end{quote}
	
	\begin{itemize}
		\item Python programming language
		\item Web application framework
		\item Authentication through Google Accounts
		\item 10GB traffic per day
		\item 1GB of data storage
	\end{itemize}
}

\section{App Engine Advert Server}
\subsection{Properties}
\frame
{
	\frametitle{(Dis)Advantages}
	\uncover<1->{Advantages:}
	\begin{itemize}
		\item <1->Powerful (distributed) database with query engine and
			transactions
		\item <1->Authentication through Google Accounts
		\item <1->Built-in frameworks (Django, WebOb, PyYAML)
	\end{itemize}

	\uncover<2->{Disadvantages:}
	\begin{itemize}
		\item <2->Python 2.5 (limited functionality)
		\item <2->HTTP(S) Only
		\item <2->Sandbox (no \texttt{fork}, no threads, no FS)
	\end{itemize}
}

\subsection{Design}
\frame
{
	\frametitle{Server Design}
	Public functions:
	\begin{itemize}
		\item <1->\texttt{add(data, metadata, path)}
		\item <1->\texttt{del(path)}
		\item <1->\texttt{get(path)}
		\item <1->\texttt{getmd(path)}
		\item <1->\texttt{find(metadata)}
	\end{itemize}
	\begin{itemize}
		\item <2->\texttt{data}: Text
		\item <2->\texttt{metadata}: Key-value pairs of Strings
		\item <2->\texttt{path}: String
	\end{itemize}
}

\frame
{
	\frametitle{Client Library}
	\begin{itemize}
		\item Written in Java
		\item Implements the GAE Advert server's protocol
		\begin{itemize}
			\item Base64 encoding
			\item JSON encoding
			\item Error handling 
		\end{itemize}
		\item Takes care of authentication
		\begin{itemize}
			\item Authentication through \emph{Google ClientLogin}
			\item Automatically refresh authentication cookie
		\end{itemize} 
	\end{itemize}
}

\section{Benchmarks}
\subsection{Server Functions}
\frame
{
	\frametitle{Add()}
	\begin{figure}[t]
	\begin{center}
	\includegraphics[trim = 0 5cm 0 5cm, width=10cm]{add_obj.pdf} 
	\caption{Adding Objects of Variable Size}
	\end{center}
	\end{figure}
}

\frame
{
	\frametitle{Add()}
	\begin{figure}[t]
	\begin{center}
	\includegraphics[trim = 0 5cm 0 5cm, width=10cm]{add_md.pdf} 
	\caption{Adding Objects with Variable Meta Data}
	\end{center}
	\end{figure}
}

\frame
{
	\frametitle{Get()}
	\begin{figure}[t]
	\begin{center}
	\includegraphics[trim = 0 5cm 0 5cm, width=10cm]{get_obj.pdf} 
	\caption{Retrieving Objects of Variable Size}
	\end{center}
	\end{figure}
}

\frame
{
	\frametitle{Get()}
	\begin{figure}[t]
	\begin{center}
	\includegraphics[trim = 0 5cm 0 5cm, width=10cm]{get_amt.pdf} 
	\caption{Retrieving Objects with Variable Meta Data}
	\end{center}
	\end{figure}
}

\frame
{
	\frametitle{Find()}
	\begin{figure}[t]
	\begin{center}
	\includegraphics[trim = 0 5cm 0 5cm, width=10cm]{find_amt.pdf} 
	\caption{Finding Meta Data}
	\end{center}
	\end{figure}
}

\subsection{Clients in Parallel}
\frame
{
	\frametitle{Parellel Benchmarks}
	\textbf{TODO:} under construction!
}

\section{Applications}
\subsection{JavaGAT}
\frame
{
	\frametitle{JavaGAT}
	\begin{quote}
		JavaGAT offers a set of coordinated, generic and flexible APIs for accessing
		grid services from application codes, portals, data managements systems, etc.
	\end{quote}
	
	\begin{itemize}
		\item File operations (File, LogicalFile, RandomAccessFile, Endpoint)
		\item File stream operations (FileInputStream, FileOutputStream)
		\item Job submission (ResourceBroker)
		\item Monitoring (Monitorable)
		\item Access to information services (AdvertService)
	\end{itemize}
}

\frame[t]
{
	\frametitle{JavaGAT structure}
	\begin{figure}[t]
	\begin{center}
	\includegraphics[width=8cm]{gat-design.png} 
	\end{center}
	\end{figure}
}

\frame[t]
{
	\frametitle{JavaGAT structure}
	\begin{figure}[t]
	\begin{center}
	\includegraphics[width=8cm]{gat-mydesign.png} 
	\end{center}
	\end{figure}
}

\frame
{
	\begin{itemize}
      \item As of yet, only a local AdvertService existed
      \item Completely transparent to user
      \item Path handling is done locally
    \end{itemize}
}

\subsection{IPL}
\frame
{
	\frametitle{IPL}
	\begin{quote}
		The IPL is a communication library is specifically designed for usage in a
		grid environment.
	\end{quote}
	\begin{figure}[h]
	\begin{center}
	\includegraphics[width=5cm]{ibis-design.png} 
	\end{center}
	\end{figure}
}

\frame[containsverbatim]
{
	\frametitle{IPL Example}
	\usebeamerfont{small}
	\begin{verbatim}
		$ $IPL_HOME/bin/ipl-server --events
		Ibis server running on 130.37.20.18-8888~bbn230
		List of Services:
		    Bootstrap service on virtual port 303
		    Central Registry service on virtual port 302
		    Management service
		Known hubs now: 130.37.20.18-8888~bbn230
		
	\end{verbatim}
	\begin{verbatim}
		$IPL_HOME/bin/ipl-run \
		-Dibis.server.address=130.37.20.18-8888~bbn230 -Dibis.pool.name=test \
		ibis.ipl.examples.Hello
		
    \end{verbatim}
}

\frame[containsverbatim]
{
	\frametitle{IPL Example Using Advert}
	\begin{verbatim}
		$ $IPL_HOME/bin/ipl-server --advert google://jondoe.appspot.com/identifier \
		--metadata author=jondoe,created=24jun,pool=test,color=purple
		Ibis server running on 130.37.20.18-8888~bbn230
		List of Services:
		    Bootstrap service on virtual port 303
		    Central Registry service on virtual port 302
		    Management service
		Known hubs now: 130.37.20.18-8888~bbn230
		
	\end{verbatim}

	\begin{verbatim}
		$IPL_HOME/bin/ipl-run \
		-Dibis.advert.address=google://jondoe.appspot.com \
		-Dibis.advert.metadata=created=24jun,color=purple \
		-Dibis.pool.name=test \
		ibis.ipl.examples.Hello
		
    \end{verbatim}
}

\section{Conclusion}
\subsection{Verdict}
\frame{
	\frametitle{Pros and Cons}
	Pros:
	\begin{itemize}
      \item[+] <1->Free of charge (until quota exceeds)
      \item[+] <1->AdvertService only existed locally
      \item[+] <1->No registry bootstrap server as of yet
    \end{itemize}
    
    \uncover<2->{Cons:}
    \begin{itemize}
      \item[-] <2->No guarantees
      \item[-] <2->\ldots
    \end{itemize}
}

\subsection{Questions}
\frame{
	\large{Any questions?}
}
\end{document}