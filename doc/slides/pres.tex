\documentclass{beamer}

\usepackage{beamerthemesplit}

\title{A Cloud-based Application Storage Service \newline for Ibis}
\author{Bas Boterman}
\date{\today}

\begin{document}

\frame{
	\titlepage
	\begin{figure}[h]
	\begin{center}
	\includegraphics[width=3cm]{msc_logo.png} 
	\end{center}
	\end{figure}
}

\section[Outline]{}
\frame{\tableofcontents}

\section{Introduction}
\subsection{Thesis}
\frame
{
  \frametitle{Thesis}
  A Cloud-based Application Storage Service for Ibis 
  \newline 
  \newline 
  \newline
  \newline 
}

\frame
{
  \frametitle{Thesis}
  A \textbf{Cloud-based} Application Storage Service for Ibis
  
  \begin{quote}
  A style of computing in which dynamically scalable resources are provided as a service over the Internet.
  \end{quote}

  E.g. the Google App Engine.
}

\frame
{
  \frametitle{Thesis}
  A Cloud-based \textbf{Application Storage Service} for Ibis
  
  \begin{quote}
  A service used by applications to store and retrieve application data.
  \end{quote}

  I.e. an Advert service.
}

\frame
{
  \frametitle{Thesis}
  A Cloud-based Application Storage Service for \textbf{Ibis}
  
  \begin{quote}
  The main goal of the Ibis project is to create an efficient Java-based platform for grid computing.
  \end{quote}

  \begin{itemize}
    \item JavaGAT (Grid Applicatin Toolkit)
    \item IPL (Ibis Portible Layer)
  \end{itemize}
}

\subsection{Google App Engine}
\frame
{
	\frametitle{Google App Engine}
	\begin{quote}
    A platform for building and hosting web applications on
	the Google infrastructure for free.
    \end{quote}
	
	\begin{itemize}
      \item Python programming language
      \item Web application framework
      \item Authentication through Google Accounts
      \item 10GB traffic per day
      \item 1GB of data storage
    \end{itemize}
}

\section{App Engine Advert Server}
\subsection{Properties}

\frame{
	\frametitle{(Dis)Advantages}
	\uncover<1->{Advantages:}
	\begin{itemize}
      \item <1->Powerful (distributed) database with query engine and
      transactions
      \item <1->Authentication through Google Accounts
      \item <1->Built-in frameworks (Django, WebOb, PyYAML)
    \end{itemize}
    
    \uncover<2->{Disadvantages:}
    \begin{itemize}
      \item <2->Python 2.5 (limited functionality)
      \item <2->HTTP(S) Only
      \item <2->Sandbox (no \texttt{fork}, no threads)
    \end{itemize}
}

\subsection{Design}
\frame{
	\frametitle{Server Design}
	Public functions:
	\begin{itemize}
      \item <1->\texttt{add(data, metadata, path)}
      \item <1->\texttt{del(path)}
      \item <1->\texttt{get(path)}
      \item <1->\texttt{getmd(path)}
      \item <1->\texttt{find(metadata)}
    \end{itemize}
    \newline
	\begin{itemize}
      \item <2->\texttt{data}: Text
      \item <2->\texttt{metadata}: Key-value pairs of Strings
      \item <2->\texttt{path}: String
    \end{itemize}
}

\subsection{Ibis}
\frame
{
	\frametitle{JavaGAT}
	\begin{quote}
    JavaGAT offers a set of coordinated, generic and flexible APIs for accessing
    grid services from application codes, portals, data managements systems, etc.
    \end{quote}
	
	\begin{itemize}
      \item File operations (File, LogicalFile, RandomAccessFile, Endpoint)
      \item File stream operations (FileInputStream, FileOutputStream)
      \item Job submission (ResourceBroker)
      \item Monitoring (Monitorable)
      \item Access to information services (AdvertService)
    \end{itemize}
}

\frame
{
	\frametitle{IPL}
	\begin{quote}
    The IPL is a communication library is specifically designed for usage in a
    grid environment.
    \end{quote}
}

\end{document}